\section{分波法}

\begin{quotation}
``我们懂得了,学物理不应该只狭窄地学一个专业。学物理应该从平地开始,
一块砖一块砖地砌,一层一层地加高。我们懂得了,抽象化应在具体的基础工作之后,而绝非在它之前。''\qquad 杨振宁
\end{quotation}


散射也叫碰撞,是研究基本粒子,原子核的主要途径。散射理论也是解释各种输运现象的基础。

\subsection{预备知识}

弹性碰撞(弹性散射):只有动能的交换,粒子内部状态(能级、自旋等)不改变。否则,称之为非弹性碰撞(非弹性散射)。

\textbf{例1:正电子入射基态氢原子}

\begin{tabular}{|l|l|}
  \hline
  % after \\: \hline or \cline{col1-col2} \cline{col3-col4} ...
  $e^ +   + H \to e^ +   + H$ & 弹性碰撞 \\
%  \hline
  $e^ +   + H \to e^ +   + H^* $ & 非弹性碰撞,$H{}^*$表示氢原子的激发态 \\
%  \hline
  $e^ +   + H \to e^ +   + e^ -   + p$ & 发应:氢原子电离 \\
%  \hline
  $e^ +   + H \to 2\gamma  + p$ & 反应:正负电子湮灭 \\
%  \hline
  $e^ +   + H \to p + \left( {e^ +  e^ -  } \right)$ & 反应:重新组合为质子和电子偶素 \\
  \hline
 \end{tabular}


\textbf{散射截面:}假设入射粒子以稳定的入射流密度$j_i$入射,由于散射中心(靶)的存在,入射粒子将被散射到不同的方向$\left( {\theta ,\varphi } \right)$出射;假设单位时间有$dn$个粒子沿$\left( {\theta ,\varphi } \right)$方向出射:



\begin{equation}\label{26-1}
dn = \sigma \left( {\theta ,\varphi } \right)j_i d\Omega
\end{equation}

$\sigma \left( {\theta ,\varphi } \right) = \frac{1}{{j_i }}\frac{{dn}}{{d\Omega }}$, 具有面积的量刚,称为散射截面\footnote{参考周士勋《量子力学教程》第176页};表示被散射到$\left( {\theta ,\varphi } \right)$方向的有效面积。

\begin{equation}\label{26-2}
\sigma _t  = \int {d\Omega \sigma \left( {\theta ,\varphi } \right)}
\end{equation}

称为总散射截面。

\textbf{平均自由程:}入射粒子入射到具有一定厚度的介质时,假设介质中单位体积存在$\rho$个散射中心(``杂质''体密度),
入射粒子经过$dx$距离后,将被$\rho Adx$个散射中心散射,总散射截面为:$\sigma _t \rho Adx$。入射粒子发生散射的几率为$P = \frac{{\sigma _t \rho Adx}}{A} = \sigma _t \rho dx$(假设各散射中心互不遮挡);

假设$n(x)$是粒子入射$x$距离后仍然不被散射的几率,由于在$dx$距离发生碰撞的几率为$\sigma _t \rho dx$,$dn(x) =  - n(x)\sigma _t \rho dx$;解出:$n(x) = n(0)\exp \left( { - \sigma _t \rho x} \right) = n(0)\exp \left( { - {x \mathord{\left/
 {\vphantom {x \lambda }} \right.
 \kern-\nulldelimiterspace} \lambda }} \right)$


其中:$\lambda  = \frac{1}{{\sigma _t \rho }}$是入射粒子平均自由程。即当$x = \lambda $时,保持原运动状态,而未被散射的粒子是$x=0$时的$\frac{1}{e}$。


\subsection{渐近行为}

取散射中心为坐标原点,$V(r)$表示散射中心与入射粒子的相互作用,薛定谔方程:



\begin{equation}\label{26-3}
\left[ { - \frac{{\hbar ^2 }}{{2m}}\nabla ^2  + V(r)} \right]\psi  = E\psi
\end{equation}

入射粒子用平面波表示,$\psi _i  = e^{ikz} $,取入射方向为$z$轴方向;($p_z  = \hbar k$,$E = \frac{{\hbar ^2 k^2 }}{{2m}}$)
由于散射中心的存在,入射波函数将发生畸变,在$r \to \infty $时可分为未被散射平面波$e^{ikz} $和被散射球面波$f\left( {\theta ,\varphi } \right)\frac{{e^{ikr} }}{r}$两部分,$f\left( {\theta ,\varphi } \right)$称为散射振幅。


如散射中心$V(r)$是中心力场,则角动量守恒;在$V(r)$作用下,波函数在$r \to \infty $时的渐近行为\footnote{入射波函数用$\psi _i  = e^{ikz} $表示时,角动量z分量为0,$l_z  = 0$,即$m = 0$,散射过程中角动量守恒,因此散射波函数$m = 0$,即散射波函数角度部分只与$\theta$有关,与$\varphi$无关,
因此$f\left( {\theta ,\varphi } \right) = f\left( \theta  \right)$}:

\begin{equation}\label{26-4}
\mathop {\lim }\limits_{r \to \infty } \psi \left( {r,\theta } \right) \to e^{ikz}  + f\left( \theta  \right)\frac{{e^{ikr} }}{r}
\end{equation}

入射粒子流密度:$j_i  = \frac{\hbar }{{2im}}\left[ {e^{ - ikz} \left( {{\textstyle{d \over {dz}}}e^{ikz} } \right) - \left( {{\textstyle{d \over {dz}}}e^{ - ikz} } \right)e^{ikz} } \right] = {{\hbar k} \mathord{\left/
                 {\vphantom {{\hbar k} m}} \right.
                 \kern-\nulldelimiterspace} m}$


被散射粒子流密度:
$j_s  = \frac{\hbar }{{2im}}\left[ {f^* \left( \theta  \right)\frac{{e^{ - ikr} }}{r}\frac{\partial }{{\partial r}}\left[ {f\left( \theta  \right)\frac{{e^{ikr} }}{r}} \right] - c.c.} \right] = \frac{{\hbar k}}{m}\frac{{\left| {f\left( \theta  \right)} \right|^2 }}{{r^2 }}$


利用:$j_s  = j_s r^2 d\Omega $

得到散射截面:

\begin{equation}\label{26-5}
\sigma \left( {\theta ,\varphi } \right) = \frac{1}{{j_i }}\frac{{dn}}{{d\Omega }} = \left| {f\left( \theta  \right)} \right|^2  = \sigma \left( \theta  \right)
\end{equation}


该式对$\theta = 0$是无意义的,因为在$\theta = 0$方向,
我们无法把散射波与未散射波(入射波)区分开,
但我们依然可以将$\theta  \ne 0$时的结果外推到$\theta = 0$时,
称为向前散射,并与总散射截面存在关系:

\begin{equation}\label{26-6}
\sigma _t  = \frac{{4\pi }}{k}{\mathop{\rm Im}\nolimits} f\left( 0 \right)
\end{equation}

公式(\ref{26-6})称为光学定理,光学定理是一个普遍关系。


$\sigma \left( \theta  \right)$是实验上可以测量的,
$f\left( \theta  \right)$可以通过理论计算求得,
因此我们可以将理论与实验进行比较。

\subsection{中心力场中的弹性散射问题:分波法}

\index{Method of partial waves: 分波法}

中心力场中薛定谔方程:

\begin{eqnarray}
% \nonumber to remove numbering (before each equation)
\left[ { - \frac{{\hbar ^2 }}{{2m}}\frac{1}{{r^2 }}\frac{\partial }{{\partial r}}\left( {r^2 \frac{\partial }{{\partial r}}} \right) + \frac{{\widehat L^2 }}{{2mr^2 }} + V(r)} \right]\psi  = E\psi \\
\widehat L^2  =  - \hbar ^2 \left[ {\frac{1}{{\sin \theta }}\frac{\partial }{{\partial \theta }}\left( {\sin \theta \frac{\partial }{{\partial \theta }}} \right) + \frac{1}{{\sin ^2 \theta }}\frac{{\partial ^2 }}{{\partial \varphi ^2 }}} \right]
 \end{eqnarray}

对应波函数是$\left( {H,L^2 ,L_z } \right)$的共同本征态;

对于散射问题:$V\left( {r \to \infty } \right) = 0$,因此我们感兴趣的是$E > 0$自由态的情形。
由于$V(r)$是中心力场,散射过程中角动量守恒,相对于散射中心有不同角动量的粒子,散射后角动量不变,
即相对于散射中心有不同$\hat L^2$本征值(角量子数)的波函数$\psi$,
在$V(r)$作用下相互独立地发生散射,散射前后波函数对应相同的$\hat L^2$本征值。
因此我们可以把入射波函数$\psi_i$用$\left( {H,L^2 ,L_z } \right)$的共同本征态$\psi_{nlm}$进行展开(分波),
各分波分别散射,可使问题简化,故称为分波法。


入射波函数用$\psi _i  = e^{ikz} $表示时,散射波函数角度部分只与$\theta$有关,与$\varphi$无关;
$\hat L^2$在$m = 0$时对应本征函数为$P_l \left( {\cos \theta } \right)$, 本征值为: $l\left( {l + 1} \right)\hbar ^2 $,
$l = 0,1,2...$


\begin{equation}\label{26-7}
\psi \left( {r,\theta } \right) = \sum\limits_{l = 0}^\infty  {R_l (r)P_l (\cos \theta )}
\end{equation}

考虑自由态波函数$E > 0$, 主量子数$n$无意义。我们把波函数分成$l = 0,1,2...$一系列分波,分别称为$s$波,$p$波,$d$波……。


径向方程可写为:

\begin{equation}\label{26-8}
\frac{{d^2 }}{{dr^2 }}R_l  + \frac{2}{r}\frac{{dR_l }}{{dr}} + \left[ {\frac{{2m\left( {E - V(r)} \right)}}{{\hbar ^2 }} - \frac{{l\left( {l + 1} \right)}}{{r^2 }}} \right]R_l  = 0, l = 0,1,2...
\end{equation}

弹性散射,能量守恒:$E = \frac{{\hbar ^2 k^2 }}{{2m}}$


为求解径向方程,引入变换:$R_l (r) = \frac{{\chi _l (r)}}{r}$


径向方程简化为:

\begin{equation}\label{26-9}
\frac{{d^2 }}{{dr^2 }}\chi _l  + \left[ {k^2  - \frac{{2mV(r)}}{{\hbar ^2 }} - \frac{{l\left( {l + 1} \right)}}{{r^2 }}} \right]\chi _l  = 0
\end{equation}


我们需要求$r \to \infty $时的渐近解,
以与$\mathop {\lim }\limits_{r \to \infty } \psi
\left( {r,\theta } \right) \to e^{ikz}  + f\left( \theta  \right)\frac{{e^{ikr} }}{r}$
比较获得$f\left( \theta  \right)$;


$r \to \infty $时,$V(r) \to 0$,$\frac{{l\left( {l + 1} \right)}}{{r^2 }} \to 0$,径向方程化为:

\begin{equation}\label{26-10}
\frac{{d^2 }}{{dr^2 }}\chi _l  + k^2 \chi _l  = 0
\end{equation}

它的解是:$\chi _l (r) \to A_l^ +  e^{ikr}  + A_l^ -  e^{ - ikr}  = A'_l \sin \left( {kr + \delta '_l } \right)$


$r \to \infty $时,$R_l (r) \to \frac{{A'_l \sin \left( {kr + \delta '_l } \right)}}{r}$


为讨论方便,定义:$A_l  = kA'_l $, $\delta _l  = \delta '_l  + \frac{l}{2}\pi $


$R_l (r) \to \frac{{A_l \sin \left( {kr - {\textstyle{l \over 2}}\pi  + \delta _l } \right)}}{{kr}}$

$\mathop {\lim }\limits_{r \to \infty } \psi \left( {r,\theta } \right) \to \sum\limits_{l = 0}^\infty  {\frac{{A_l }}{{kr}}\sin \left( {kr - {\textstyle{l \over 2}}\pi  + \delta _l } \right)P_l (\cos \theta )} $

将此式与$\mathop {\lim }\limits_{r \to \infty } \psi \left( {r,\theta } \right) \to e^{ikz}  + f\left( \theta  \right)\frac{{e^{ikr} }}{r}$
比较就可获得$f\left( \theta  \right)$;


为此,我们需将$e^{ikz} $按$P_l \left( {\cos \theta } \right)$进行展开:


\begin{equation}\label{26-11}
e^{ikz}  = e^{ikr\cos \theta }  = \sum\limits_{l = 0}^\infty  {\left( {2l + 1} \right)i^l j_l \left( {kr} \right)P_l \left( {\cos \theta } \right)}
\end{equation}

其中$j_l \left( {kr} \right)$是球贝塞尔函数;它的渐近表示式为

$j_l \left( {kr} \right) = \mathop {\lim }\limits_{r \to \infty } \sqrt {\frac{\pi }{{2kr}}} J_{l + {\textstyle{1 \over 2}}} \left( {kr} \right) \to \frac{1}{{kr}}\sin \left( {kr - {\textstyle{l \over 2}}\pi } \right)$


两种渐近表示应在$r \to \infty $时相等,即:

$\sum\limits_{l = 0}^\infty  {\left( {2l + 1} \right)i^l \frac{1}{{kr}}\sin \left( {kr - {\textstyle{l \over 2}}\pi } \right)P_l \left( {\cos \theta } \right)}  + \frac{{f\left( \theta  \right)}}{r}e^{ikr}  = \sum\limits_{l = 0}^\infty  {\frac{{A_l }}{{kr}}\sin \left( {kr - {\textstyle{l \over 2}}\pi  + \delta _l } \right)P_l (\cos \theta )} $


利用:$\sin \alpha  = \frac{1}{{2i}}\left( {e^{i\alpha }  - e^{ - i\alpha } } \right)$


$\begin{array}{l}
 \sum\limits_{l = 0}^\infty  {\left( {2l + 1} \right)i^{l - 1} \frac{1}{{2kr}}\left[ {e^{i\left( {kr - {\textstyle{l \over 2}}\pi } \right)}  - e^{ - i\left( {kr - {\textstyle{l \over 2}}\pi } \right)} } \right]P_l \left( {\cos \theta } \right)}  + \frac{{f\left( \theta  \right)}}{r}e^{ikr}  \\
  = \sum\limits_{l = 0}^\infty  {\frac{{A_l }}{{kr}}\frac{1}{{2i}}\left[ {e^{i\left( {kr - {\textstyle{l \over 2}}\pi  + \delta _l } \right)}  - e^{ - i\left( {kr - {\textstyle{l \over 2}}\pi  + \delta _l } \right)} } \right]P_l (\cos \theta )}  \\
 \end{array}$





得到:

$\begin{array}{l}
 \left[ {\frac{{f\left( \theta  \right)}}{r} + \sum\limits_{l = 0}^\infty  {\left( {2l + 1} \right)i^{l - 1} \frac{{e^{ - i{\textstyle{l \over 2}}\pi } }}{{2kr}}P_l \left( {\cos \theta } \right)} } \right]e^{ikr}  - \left[ {\sum\limits_{l = 0}^\infty  {\left( {2l + 1} \right)i^{l - 1} \frac{{e^{i{\textstyle{l \over 2}}\pi } }}{{2kr}}P_l \left( {\cos \theta } \right)} } \right]e^{ - ikr}  \\
  = \left[ {\sum\limits_{l = 0}^\infty  {\frac{{A_l }}{{kr}}\frac{{e^{i\left( {\delta _l  - {\textstyle{l \over 2}}\pi } \right)} }}{{2i}}P_l (\cos \theta )} } \right]e^{ikr}  - \left[ {\sum\limits_{l = 0}^\infty  {\frac{{A_l }}{{kr}}\frac{{e^{ - i\left( {\delta _l  - {\textstyle{l \over 2}}\pi } \right)} }}{{2i}}P_l (\cos \theta )} } \right]e^{ - ikr}  \\
 \end{array}$


两边乘以$2ikr$:

$\begin{array}{l}
 \left[ {2kif\left( \theta  \right) + \sum\limits_{l = 0}^\infty  {\left( {2l + 1} \right)i^l e^{ - i{\textstyle{l \over 2}}\pi } P_l \left( {\cos \theta } \right)} } \right]e^{ikr}  - \left[ {\sum\limits_{l = 0}^\infty  {\left( {2l + 1} \right)i^l e^{i{\textstyle{l \over 2}}\pi } P_l \left( {\cos \theta } \right)} } \right]e^{ - ikr}  \\
  = \left[ {\sum\limits_{l = 0}^\infty  {A_l e^{i\left( {\delta _l  - {\textstyle{l \over 2}}\pi } \right)} P_l (\cos \theta )} } \right]e^{ikr}  - \left[ {\sum\limits_{l = 0}^\infty  {A_l e^{ - i\left( {\delta _l  - {\textstyle{l \over 2}}\pi } \right)} P_l (\cos \theta )} } \right]e^{ - ikr}  \\
 \end{array}$


因此:

$\left\{ \begin{array}{l}
 2kif\left( \theta  \right) + \sum\limits_{l = 0}^\infty  {\left( {2l + 1} \right)i^l e^{ - i{\textstyle{l \over 2}}\pi } P_l \left( {\cos \theta } \right)}  = \sum\limits_{l = 0}^\infty  {A_l e^{i\left( {\delta _l  - {\textstyle{l \over 2}}\pi } \right)} P_l (\cos \theta )}  \\
 \sum\limits_{l = 0}^\infty  {\left( {2l + 1} \right)i^l e^{i{\textstyle{l \over 2}}\pi } P_l \left( {\cos \theta } \right)}  = \sum\limits_{l = 0}^\infty  {A_l e^{ - i\left( {\delta _l  - {\textstyle{l \over 2}}\pi } \right)} P_l (\cos \theta )}  \\
 \end{array} \right.$


利用勒让德多项式的正交性:
$\int_0^\pi  {P_l \left( {\cos \theta } \right)P_{l'} \left( {\cos \theta } \right)} \sin \theta d\theta  = \frac{2}{{2l + 1}}\delta _{ll'} $


由:$\sum\limits_{l = 0}^\infty  {\left( {2l + 1} \right)i^l e^{i{\textstyle{l \over 2}}\pi } P_l \left( {\cos \theta } \right)}  = \sum\limits_{l = 0}^\infty  {A_l e^{ - i\left( {\delta _l  - {\textstyle{l \over 2}}\pi } \right)} P_l (\cos \theta )} $


得到:$A_l  = \left( {2l + 1} \right)i^l e^{i\delta _l } $;并利用:$i^l  = e^{i{\textstyle{l \over 2}}\pi } $


$\begin{array}{l}
 2kif\left( \theta  \right) = \sum\limits_{l = 0}^\infty  {\left[ {\left( {2l + 1} \right)\left( {e^{i2\delta _l }  - 1} \right)} \right]P_l (\cos \theta )}  \\ =  \sum\limits_{l = 0}^\infty  {\left( {2l + 1} \right)\left( {\cos 2\delta _l  + i\sin 2\delta _l  - 1} \right)P_l (\cos \theta )}  \\
  = \sum\limits_{l = 0}^\infty  {\left( {2l + 1} \right)\left( {1 - 2\sin ^2 \delta _l  + 2i\sin \delta _l \cos \delta _l  - 1} \right)P_l (\cos \theta )}  \\
  = \sum\limits_{l = 0}^\infty  {\left( {2l + 1} \right)\left[ {2i\sin \delta _l \left( {\cos \delta _l  + i\sin \delta _l } \right)} \right]P_l (\cos \theta )}  \\
  = \sum\limits_{l = 0}^\infty  {\left( {2l + 1} \right)P_l (\cos \theta )2ie^{i\delta _l } \sin \delta _l }  \\
 \end{array}$


所以:


\begin{equation}\label{26-12}
f\left( \theta  \right) = \frac{1}{k}\sum\limits_{l = 0}^\infty  {\left( {2l + 1} \right)P_l (\cos \theta )e^{i\delta _l } \sin \delta _l }
\end{equation}


可见散射振幅$f(\theta)$的求解归结为各分波相移$\delta_l$的求解.



散射截面:$\sigma \left( \theta  \right) = \left| {f\left( \theta  \right)} \right|^2 $



总散射截面:

$\sigma _t  = 2\pi \int_0^\pi  {d\theta \sin \theta \left| {f\left( \theta  \right)} \right|^2 }  = ... = \frac{{4\pi }}{{k^2 }}\sum\limits_{l = 0}^\infty  {\left( {2l + 1} \right)\sin ^2 \delta _l }  = \sum\limits_{l = 0}^\infty  {\sigma _l } $



其中:$\sigma _l  = \frac{{4\pi }}{{k^2 }}\left( {2l + 1} \right)\sin ^2 \delta _l $
是第$l$个分波的散射截面。



当$\theta = 0$时,向前散射:




$f\left( 0 \right) = \frac{1}{k}\sum\limits_{l = 0}^\infty  {\left( {2l + 1} \right)P_l (1)e^{i\delta _l } \sin \delta _l }  = \frac{1}{k}\sum\limits_{l = 0}^\infty  {\left( {2l + 1} \right)\left( {\sin \delta _l \cos \delta _l  + i\sin ^2 \delta _l } \right)}$


因此验证了光学定理:$\sigma _t  = \frac{{4\pi }}{k}{\mathop{\rm Im}\nolimits} f\left( 0 \right)$


分波法适用范围:原则上分波法是处理散射问题的普遍方法,
但只有随$l$增加,级数求和$\sigma _t  = \sum\limits_{l = 0}^\infty
{\sigma _l }
$收敛快的情形才有意义。分波法在低能散射情况下最为适用\footnote{参考周士勋《量子力学教程》
第183页}。


\subsection{低能粒子受球方势阱/垒散射}


\begin{equation}\label{26-13}
V(r) = \left\{ \begin{array}{l}
 V_0 ,r \le a \\
 0,r > a \\
 \end{array} \right.
\end{equation}


在势阱情况下$V_0  =  - \left| {V_0 } \right| < 0$,低能情形$ka \ll 1$,只考虑$s$波($l=0$)。

径向方程:$\frac{{d^2 }}{{dr^2 }}\chi  + \left[ {k^2  - V(r)} \right]\chi  = 0$, 即:


\begin{center}
$\left\{ \begin{array}{l}
 \frac{{d^2 }}{{dr^2 }}\chi  + k'^2 \chi  = 0,r \le a \\
 \frac{{d^2 }}{{dr^2 }}\chi  + k^2 \chi  = 0,r > a \\
 \end{array} \right.$
\end{center}



这里:$k'^2  = k^2  + \frac{{2m\left| {V_0 } \right|}}{{\hbar ^2 }}$


利用边界条件$\chi (r = 0) = 0$, 解出:


\begin{center}
$\left\{ \begin{array}{l}
 \chi (r) = A\sin k'r,r \le a \\
 \chi (r) = B\sin \left( {kr + \delta _0 } \right),r > a \\
 \end{array} \right.$
\end{center}

利用$r=a$时,连续性条件:

\begin{eqnarray*}
\left\{ \begin{array}{l}
 A\sin k'a = B\sin \left( {ka + \delta _0 } \right) \\
 Ak'\cos k'a = Bk\cos \left( {ka + \delta _0 } \right) \\
 \end{array} \right.
 \end{eqnarray*}

解出:$\frac{1}{{k'}}\tan k'a = \frac{1}{k}\tan \left( {ka + \delta _0 } \right)$


相移:$\delta _0  = \arctan \left[ {\frac{k}{{k'}}\tan k'a} \right] - ka$


利用低能近似条件:$k \to 0$,

\begin{center}
$\sin \delta _0  \approx \tan \delta _0  \approx \delta _0  \approx \frac{k}{{k'}}\tan k'a - ka$
\end{center}


利用:$k'^2  = k^2  + \frac{{2m\left| {V_0 } \right|}}{{\hbar ^2 }}$,$k' \approx \frac{{\sqrt {2m\left| {V_0 } \right|} }}{\hbar } = k_0 $


\begin{eqnarray*}
\sin \delta _0  \approx \delta _0  \approx \frac{k}{{k'}}\tan k'a - ka \approx ka\left[ {\frac{{\tan k_0 a}}{{k_0 a}} - 1} \right]
 \end{eqnarray*}

\begin{center}
$\sigma _0  = \frac{{4\pi }}{{k^2 }}\sin ^2 \delta _0  = 4\pi a^2 \left( {\frac{{\tan k_0 a}}{{k_0 a}} - 1} \right)^2$
\end{center}


类似地对势垒:$V_0  = \left| {V_0 } \right| > 0$,$k_0  = i\frac{{\sqrt {2m\left| {V_0 } \right|} }}{\hbar } = i\left| {k_0 } \right|$

$\tan i\left| {k_0 } \right|a = i\frac{{e^{\left| {k_0 } \right|a}  - e^{ - \left| {k_0 } \right|a} }}{{e^{\left| {k_0 } \right|a}  + e^{ - \left| {k_0 } \right|a} }} = i\tanh \left| {k_0 } \right|a$


$\sigma _0  = \frac{{4\pi }}{{k^2 }}\sin ^2 \delta _0  = 4\pi a^2 \left( {\frac{{\tanh \left| {k_0 } \right|a}}{{\left| {k_0 } \right|a}} - 1} \right)^2 $



对于刚球散射\footnote{刚球散射经典情形,总散射截面$\sigma _t  = \pi a^2 $}:$V_0  \to \infty $,$\left| {k_0 } \right|a \to \infty $,$\tanh \left| {k_0 } \right|a \to 1$

所以:

\begin{center}
$\sigma _0  = \frac{{4\pi }}{{k^2 }}\sin ^2 \delta _0  = 4\pi a^2 \left( {\frac{{\tanh \left| {k_0 } \right|a}}{{\left| {k_0 } \right|a}} - 1} \right)^2  \to 4\pi a^2$
\end{center}